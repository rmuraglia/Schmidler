\abstract

Free energy calculations are a computational method for determining thermodynamic quantities, such as free energies of binding, via simulation. 
Currently, due to computational and algorithmic limitations, free energy calculations are limited in scope.
% Efficient free energy calculations would serve as an effective tool for computational molecular design, obviating the need for a significant amount of costly and time consuming synthesis and experiment.
In this work, we propose two methods for improving the efficiency of free energy calculations.
First, we expand the state space of alchemical intermediates, and show that this expansion enables us to calculate free energies along lower variance paths.
We use Q-learning, a reinforcement learning technique, to discover and optimize paths at low computational cost.
Second, we reduce the cost of sampling along a given path by using sequential Monte Carlo samplers.
We develop a new free energy estimator, pCrooks (pairwise Crooks), a variant on the Crooks fluctuation theorem (CFT), which enables decomposition of the variance of the free energy estimate for discrete paths, while retaining beneficial characteristics of CFT.
Combining these two advancements, we show that for some test models, optimal expanded-space paths have a nearly 80\% reduction in variance relative to the standard path.
Additionally, our free energy estimator converges at a more consistent rate and on average 1.8 times faster when we enable path searching, even when the cost of path discovery and refinement is considered.

